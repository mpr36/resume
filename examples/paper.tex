\documentclass{article}

% Packages
\usepackage[utf8]{inputenc} % Handle UTF-8 input
\usepackage{amsmath}        % For mathematical equations
\usepackage{graphicx}       % For including images
\usepackage{hyperref}       % For hyperlinks

\title{Sample \LaTeX\ Document}
\author{Your Name}
\date{\today}

\begin{document}

% Title and Author
\maketitle

% Abstract
\begin{abstract}
This is a sample \LaTeX\ document. It includes sections, lists, tables, images, and mathematical equations.
\end{abstract}

% Section 1: Introduction
\section{Introduction}
\LaTeX\ is a high-quality typesetting system; it includes features designed for the production of technical and scientific documentation. \LaTeX\ is the de facto standard for the communication and publication of scientific documents.

% Section 2: Lists
\section{Lists}
You can create itemized or enumerated lists easily in \LaTeX.

% Itemized List
\subsection{Itemized List}
\begin{itemize}
    \item First item
    \item Second item
    \item Third item
\end{itemize}

% Enumerated List
\subsection{Enumerated List}
\begin{enumerate}
    \item First item
    \item Second item
    \item Third item
\end{enumerate}

% Section 3: Table
\section{Table}
Here is a simple table:

\begin{table}[h!]
\centering
\begin{tabular}{|c|c|c|}
\hline
\textbf{Column 1} & \textbf{Column 2} & \textbf{Column 3} \\ \hline
A                & B                 & C                 \\ \hline
D                & E                 & F                 \\ \hline
G                & H                 & I                 \\ \hline
\end{tabular}
\caption{A sample table.}
\end{table}

% Section 4: Math Equations
\section{Math Equations}
Mathematics is easy in \LaTeX. Here is an inline equation: $E = mc^2$.

You can also write displayed equations:
\begin{equation}
\int_0^\infty e^{-x^2} \, dx = \frac{\sqrt{\pi}}{2}
\end{equation}

% Section 5: Including Images
\section{Including Images}
You can include images using the \texttt{graphicx} package. 

% Section 6: Hyperlinks
\section{Hyperlinks}
You can add hyperlinks in your document using the \texttt{hyperref} package. Here is a link to \href{https://www.latex-project.org}{The \LaTeX\ Project}.

% Section 7: Conclusion
\section{Conclusion}
This is a basic \LaTeX\ document. You can expand it by adding more sections, figures, tables, or equations as needed.

\end{document}
